\section{Motivation}

\label{sec:intro:motivation}

Die Entwicklung von Quantencomputern schreitet voran. Wenn Quantencomputer jedoch einsatzfähig sind, wird es mit ihnen möglich sein, viele der aktuell verwendeten kryptographischen Algorithmen zu brechen. Dies liegt daran, dass diese auf Berechnungen basieren, die auf klassischen Computern nicht in einer vertretbaren Zeit durchführbar sind. Dies sind vor allem die Primfaktorzerlegung und der diskrete Logarithmus. Quantencomputer können diese Berechnungen aufgrund ihrer Eigenschaften jedoch schneller durchführen. Davon sind auch die im TLS 1.3 Handshake verwendeten Algorithmen betroffen.\\

Um zu verhindern, dass hierdurch Probleme entstehen, müssen rechtzeitig PQC-Algorithmen identifiziert werden, mit denen die dann unsicherern Algorithmen ersetzt werden können. Dabei ist zu beachten, dass diese andere Charakteristika haben. Damit der Einsatz von PQC-Algorithmen gelingen kann, muss bekannt sein, wie sich diese unter verschiedenen Netzwerkbedingungen verhalten.