\chapter{Einleitung}

\label{ch:intro}

In TLS 1.3 wird ein Handshake durchgeführt, bei dem ein Schlüsselaustausch sowie die Authentifizierung des Servers gegenüber dem Client durchgeführt wird. Bei Bedarf kann auch eine beidseitige Authentifizierung durchgeführt werden. Erst im Anschluss findet die eigentliche Kommunikation statt. Aktuell werden dabei für die Authentifizierung die klassischen asymmetrischen Verfahren EdDSA, ECDSA und RSA verwendet. In dieser Arbeit soll betrachtet werden, inwiefern sich die Performanz von TLS 1.3 verändert, wenn stattdessen PQC-Verfahren, d. h. kryptographische Verfahren, die gegen Quantencomputer resistent sind, eingesetzt werden.